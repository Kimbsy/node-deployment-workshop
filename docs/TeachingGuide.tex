% Thomas Mortensson - September 2013

\documentclass[twocolumn]{article}
\usepackage[margin=0.5in]{geometry}
\usepackage{listings}
\usepackage{hyperref}
\usepackage[usenames,dvipsnames]{color}
\usepackage{booktabs}

\renewcommand{\familydefault}{\sfdefault}
\definecolor{DarkGreen}{rgb}{0.0,0.4,0.0} % Comment color
\definecolor{highlight}{RGB}{255,251,204} % Code highlight color

\lstdefinestyle{Style1}{ % Define a style for your code snippet, multiple definitions can be made if, for example, you wish to insert multiple code snippets using different programming languages into one document
language=Python, % Detects keywords, comments, strings, functions, etc for the language specified
backgroundcolor=\color{highlight}, % Set the background color for the snippet - useful for highlighting
basicstyle=\footnotesize\ttfamily, % The default font size and style of the code
breakatwhitespace=false, % If true, only allows line breaks at white space
breaklines=true, % Automatic line breaking (prevents code from protruding outside the box)
captionpos=b, % Sets the caption position: b for bottom; t for top
commentstyle=\usefont{T1}{pcr}{m}{sl}\color{DarkGreen}, % Style of comments within the code - dark green courier font
deletekeywords={}, % If you want to delete any keywords from the current language separate them by commas
%escapeinside={\%}, % This allows you to escape to LaTeX using the character in the bracket
firstnumber=1, % Line numbers begin at line 1
frame=single, % Frame around the code box, value can be: none, leftline, topline, bottomline, lines, single, shadowbox
frameround=tttt, % Rounds the corners of the frame for the top left, top right, bottom left and bottom right positions
keywordstyle=\color{Blue}\bf, % Functions are bold and blue
morekeywords={}, % Add any functions no included by default here separated by commas
numbers=left, % Location of line numbers, can take the values of: none, left, right
numbersep=10pt, % Distance of line numbers from the code box
numberstyle=\tiny\color{Gray}, % Style used for line numbers
rulecolor=\color{black}, % Frame border color
showstringspaces=false, % Don't put marks in string spaces
showtabs=false, % Display tabs in the code as lines
stepnumber=5, % The step distance between line numbers, i.e. how often will lines be numbered
stringstyle=\color{Purple}, % Strings are purple
tabsize=2, % Number of spaces per tab in the code
}

\begin{document}
\lstset{style=Style1}

\title{Node Deployment Workshop} 
\author{Thomas Mortensson and Graham Laming\\
        	Computer Science Department,\\
		University of Bristol,\\
		\texttt{\href{mailto:tm0797@bristol.ac.uk}{tm0797@bristol.ac.uk}, \href{mailto:gl1646@bristol.ac.uk}{gl1646@bristol.ac.uk}} 
		}
\date{\today} 
\maketitle

\section{Introduction to Node - What is it?}

Node.js is a platform built on Chrome's JavaScript runtime for easily building fast, scalable network applications. Node.js uses an event-driven, non-blocking I/O model that makes it lightweight and efficient, perfect for data-intensive real-time applications that run across distributed devices. One such application could be a socket based reactive chat application. In this deployment session we will aim to build and deploy a Node.js based web application using Socket.IO on top of the brilliant \href{http://koding.com}{koding.com} hosting environment. With the skills gained from this workshop you should be equipped to go out to build and deploy your own reactive web applications on a hosted environment such as Koding or in Cloud environments such as DigitalOcean or locally on your own Raspberry Pi!

\section {Getting the source}

All content explained in this workshop is contained in an easily deployable repository online. This can be accessed at:\\
\\
{\begin{center}
\url{https://github.com/thomasmortensson/node-deployment-workshop}\\
\end{center}
}

You can browse through this repository at your own leisure if you wish to go back over things we have covered in the workshop or just wish to have a simple base application to start with. Using the instructions in the README.md file you should be able to easily deploy this application on \href{http://koding.com}{koding.com} or in a \href{http://digitalocean.com}{DigitalOcean} droplet.

\section{Deployment}

\subsection{ubuntu and apt-get}

\subsection{Node and npm}

\subsection{Bower}







\end{document} 